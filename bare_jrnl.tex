%% bare_jrnl.tex
%% V1.4b
%% 2015/08/26
%% by Michael Shell
%% see http://www.michaelshell.org/
%% for current contact information.
%%
%% This is a skeleton file demonstrating the use of IEEEtran.cls
%% (requires IEEEtran.cls version 1.8b or later) with an IEEE
%% journal paper.
%%
%% Support sites:
%% http://www.michaelshell.org/tex/ieeetran/
%% http://www.ctan.org/pkg/ieeetran
%% and
%% http://www.ieee.org/

%%*************************************************************************
%% Legal Notice:
%% This code is offered as-is without any warranty either expressed or
%% implied; without even the implied warranty of MERCHANTABILITY or
%% FITNESS FOR A PARTICULAR PURPOSE! 
%% User assumes all risk.
%% In no event shall the IEEE or any contributor to this code be liable for
%% any damages or losses, including, but not limited to, incidental,
%% consequential, or any other damages, resulting from the use or misuse
%% of any information contained here.
%%
%% All comments are the opinions of their respective authors and are not
%% necessarily endorsed by the IEEE.
%%
%% This work is distributed under the LaTeX Project Public License (LPPL)
%% ( http://www.latex-project.org/ ) version 1.3, and may be freely used,
%% distributed and modified. A copy of the LPPL, version 1.3, is included
%% in the base LaTeX documentation of all distributions of LaTeX released
%% 2003/12/01 or later.
%% Retain all contribution notices and credits.
%% ** Modified files should be clearly indicated as such, including  **
%% ** renaming them and changing author support contact information. **
%%*************************************************************************


% *** Authors should verify (and, if needed, correct) their LaTeX system  ***
% *** with the testflow diagnostic prior to trusting their LaTeX platform ***
% *** with production work. The IEEE's font choices and paper sizes can   ***
% *** trigger bugs that do not appear when using other class files.       ***                          ***
% The testflow support page is at:
% http://www.michaelshell.org/tex/testflow/



\documentclass[journal]{IEEEtran}
\usepackage{amsmath}
\usepackage{amssymb}
\usepackage{algorithmic,color}

\usepackage{algorithm}
\newcommand {\empstr} {e^{i\theta}}
\def\otu{u}
\def\ots{s}

\renewcommand{\algorithmicrequire}{\textbf{Input:}}
\renewcommand{\algorithmicensure}{\textbf{Output:}}
\newcommand {\qcf}[1] {{\sf{#1}}}

\newcommand {\qc}[1] {{\sf{#1}}}
\def\>{\ensuremath{\rangle}}
\def\<{\ensuremath{\langle}}
\def\sl {\ensuremath{\llparenthesis}}
\def\sr{\ensuremath{\rrparenthesis}}
\def\-{\ensuremath{\textrm{-}}}
\def\ott{t}

\def\comm{\ensuremath{\leftrightarrow^*}}
\def\reach{\ensuremath{\rightarrow^*}}


\def\ctp{P}
\def\ctq{Q}

\def\fdmu{\Delta}
\def\fdnu{\dnu}
\def\fdomega{\domega}

\def\dmu{\mu}
\def\dnu{\nu}
\def\domega{\omega}

\def\fpi{\widehat{\pi}}
\def\h{\ensuremath{\mathcal{H}}}
\def\p{\ensuremath{\mathcal{P}}}
\def\l{\ensuremath{\mathcal{L}}}
\def\g{\ensuremath{\mathcal{G}}}
\def\lh{\ensuremath{\mathcal{L(H)}}}
\def\dh{\ensuremath{\mathcal{D(H})}}
\def\q{\bold Q}
\def\Q{\ensuremath{\mathbb Q}}
\def\P{\ensuremath{\mathbb P}}
\def\SO{\ensuremath{\mathcal{SO}}}
\def\HP{\ensuremath{\mathcal{HP}}}
\def\hpe{\ensuremath{\mathcal{\e}}}

\def\r{\ensuremath{\mathcal{R}}}
\def\R{\ensuremath{\mathfrak{R}}}
\def\f{\ensuremath{\mathcal{F}}}
\def\m{\ensuremath{\mathcal{M}}}
\def\u{\ensuremath{\mathcal{U}}}
\def\k{\ensuremath{\mathcal{K}}}
\def\K{\ensuremath{\mathfrak{K}}}
\def\S{\ensuremath{\mathfrak{S}}}
\def\s{\ensuremath{\mathcal{S}}}
\def\t{\ensuremath{\mathcal{T}}}
\def\u{\ensuremath{\mathcal{U}}}
\def\U{\ensuremath{\mathfrak{U}}}
\def\L{\ensuremath{\mathfrak{L}}}
\def\x{\ensuremath{\mathcal{X}}}
\def\y{\ensuremath{\mathcal{Y}}}
\def\z{\ensuremath{\mathcal{Z}}}
\def\v{\ensuremath{\mathcal{V}}}

\def\st{\ensuremath{\mathfrak{t}}}
\def\su{\ensuremath{\mathfrak{u}}}
\def\ss{\ensuremath{\mathfrak{s}}}


\def\ra{\ensuremath{\rightarrow}}
\def\a{\ensuremath{\mathcal{A}}}
\def\b{\ensuremath{\mathcal{B}}}
\def\c{\ensuremath{\mathcal{C}}}

\def\e{\ensuremath{\mathcal{E}}}
\def\f{\ensuremath{\mathcal{F}}}
\def\l{\ensuremath{\mathcal{L}}}
\def\X{\mbox{\bf{X}}}
\def\N{\mathbb{N}}
\def\sreal{\mathbb{R}}

\newcommand {\while} {\mbox{\bf{while}}}
\newcommand {\ddo} {\mbox{\bf{do}}}
\newcommand {\pend} {\mbox{\bf{end}}}


\def\d{\ensuremath{\mathcal{D}}}
\def\dh{\ensuremath{\mathcal{D(H)}}}
\def\lh{\ensuremath{\mathcal{L(H)}}}
\def\le{\ensuremath{\sqsubseteq}}
\def\ge{\ensuremath{\sqsupseteq}}
\def\eval{\ensuremath{{\psi}}}
\def\aeq{\ensuremath{{\ \equiv\ }}}
%\def\snt{\ensuremath{\sl \ott, \e\sr}}
\def\osnt{\ensuremath{\sl \ott, \e\sr}}
\def\snt{\st}
\def\snti{\ensuremath{\sl \ott_i, \e_i\sr}}
%\def\snu{\ensuremath{\sl \otu, \f\sr}}
\def\osnu{\ensuremath{\sl \otu, \f\sr}}
\def\osns{\ensuremath{\sl s, \g\sr}}
\def\snu{\su}
\def\fdist{\ensuremath{\d ist_\h}}
\def\dist{\ensuremath{Dist}}
\def\wtx{\ensuremath{\widetilde{X}}}

\def\bv{1{v}}
\def\bV{\mathbf{V}}
\def\bf{\mathbf{f}}
\def\bw{\mathbf{w}}
\def\zo{\mathbf{0}}
\def\bX{\mathbf{X}}
\def\bDelta{\mathbf{\Delta}}
\def\bdelta{\boldsymbol{\delta}}
\def\next{\mathcal{X}}
\def\until{\mathcal{U}}

\def\leqI{\ensuremath{\mathcal{SI}(\h)}}
\def\leqIq{\ensuremath{\mathcal{SI}_{\eqsim}(\h)}}
\def\oact{\ensuremath{\alpha}}
\def\oactb{\ensuremath{\beta}}
\def\sact{\ensuremath{\gamma}}
\def\fpi{\ensuremath{\widehat{\pi}}}
\newcommand{\supp}[1]{\ensuremath{\lceil{#1}\rceil}}
\newcommand{\support}[1]{\lceil{#1}\rceil}

\newcommand{\abis}{\stackrel{e^{i\theta}}\approx}
\newcommand{\abisa}[1]{\stackrel{#1}\approx}
\newcommand {\qbit} {\mbox{\bf{new}}}
\newcommand {\nil} {\mbox{\bf{nil}}}
\newcommand {\iif} {\mbox{\bf{if}}}
\newcommand {\then} {\mbox{\bf{then}}}
\newcommand {\eelse} {\mbox{\bf{else}}}
\newcommand {\true} {\mbox{\texttt{true}}}
\newcommand {\false} {\mbox{\texttt{false}}}
\renewcommand{\theenumi}{(\arabic{enumi})}
\renewcommand{\labelenumi}{\theenumi}
\newcommand{\tr}{{\rm tr}}
\newcommand{\rto}[1]{\stackrel{#1}\rightarrow}
\newcommand{\orto}[1]{\stackrel{#1}\longrightarrow}
\newcommand{\srto}[1]{\stackrel{#1}\longmapsto}
\newcommand{\sRto}[1]{\stackrel{#1}\Longmapsto}


%\newcommand{\rrto}[1]{\stackrel{#1}\hookrightarrow}
\newcommand{\rrto}[1]{\xhookrightarrow{#1}}
%\newcommand{\con}[2]{{#1}\triangleleft b \triangleright {#2}}
\newcommand{\con}[3]{\iif\ {#1}\ \then\ {#2}\ \eelse\ {#3}}

\newcommand{\Rto}[1]{\stackrel{#1}\Longrightarrow}
\newcommand{\nrto}[1]{\stackrel{#1}\nrightarrow}

\newcommand{\Rhto}[1]{\stackrel{\widehat{#1}}\Longrightarrow}
\newcommand{\define}{\stackrel{definition}=}
\newcommand{\rsim}{\simeq}
\newcommand{\obis}{\approx_o}
\newcommand{\sbis}{\ \dot\approx\ }
\newcommand{\stbis}{\ \dot\sim\ }
\newcommand{\nssbis}{\ \dot\nsim\ }

\newcommand{\bis}{\sim}
\newcommand{\rat}{\rightarrowtail}
\newcommand{\wbis}{\approx}
\newcommand{\id}{\mathcal{I}}
\newcommand{\stet}[1]{\{ {#1}  \}  } % singleton set
%\newcommand{\ar}[1]{\mathrel{\stackrel{#1}{\longrightarrow}}}
\newcommand{\unw}[1]{\stackrel{{#1}}\sim}
\newcommand{\rma}[1]{\stackrel{{#1}}\approx}

\def\step{\textsf{step}}
\def\obs{\textsf{obs}}
\def\dom{\textsf{dom}}
\def\purge{\textsf{ipurge}}
\def\source{\textsf{sources}}
\def\cnt{\textsf{cnt}}
\def\read{\textsf{read}}
\def\alter{\textsf{alter}}
\def\dirac#1{\delta_{#1}}


%from Yuan
\def\<{\langle}
\def\>{\rangle}
\def\l{\mathcal{L}}
\def\k{\mathcal{K}}
\def\E{\mathcal{E}}
\def\G{\mathcal{G}}
\def\H{\mathcal{H}}
\def\R{\mathcal{R}}
\def\supp{\textnormal{supp}}
\def\qmc {\color{red}}
\def\dtmc {\color{black}}
\newcommand{\ysim}[1]{\stackrel{#1}\sim}
\def\z{\mathbf{0}}
\newcommand{\TRANDA}[3]{#1\xrightarrow{#2}_{{\sf D}}#3}
\def\pdist{\mathit{pDist}}
\def\supp{\mathit{supp}}
\newtheorem{theorem}{Theorem}
\newtheorem{proposition}{Proposition}
\newtheorem{corollary}{Corollary}
\newtheorem{lemma}{Lemma}
\newtheorem{remark}{Remark}
\newtheorem{definition}{Definition}
\newtheorem{example}{Example}
% If IEEEtran.cls has not been installed into the LaTeX system files,
% manually specify the path to it like:
% \documentclass[journal]{../sty/IEEEtran}

\newcommand{\authorComment}[3]{\color{#1}#2: {#3} :#2\color{black}}

\newcommand{\yf}[1]{\authorComment{blue}{YF}{#1}}



% Some very useful LaTeX packages include:
% (uncomment the ones you want to load)


% *** MISC UTILITY PACKAGES ***
%
%\usepackage{ifpdf}
% Heiko Oberdiek's ifpdf.sty is very useful if you need conditional
% compilation based on whether the output is pdf or dvi.
% usage:
% \ifpdf
%   % pdf code
% \else
%   % dvi code
% \fi
% The latest version of ifpdf.sty can be obtained from:
% http://www.ctan.org/pkg/ifpdf
% Also, note that IEEEtran.cls V1.7 and later provides a builtin
% \ifCLASSINFOpdf conditional that works the same way.
% When switching from latex to pdflatex and vice-versa, the compiler may
% have to be run twice to clear warning/error messages.






% *** CITATION PACKAGES ***
%
%\usepackage{cite}
% cite.sty was written by Donald Arseneau
% V1.6 and later of IEEEtran pre-defines the format of the cite.sty package
% \cite{} output to follow that of the IEEE. Loading the cite package will
% result in citation numbers being automatically sorted and properly
% "compressed/ranged". e.g., [1], [9], [2], [7], [5], [6] without using
% cite.sty will become [1], [2], [5]--[7], [9] using cite.sty. cite.sty's
% \cite will automatically add leading space, if needed. Use cite.sty's
% noadjust option (cite.sty V3.8 and later) if you want to turn this off
% such as if a citation ever needs to be enclosed in parenthesis.
% cite.sty is already installed on most LaTeX systems. Be sure and use
% version 5.0 (2009-03-20) and later if using hyperref.sty.
% The latest version can be obtained at:
% http://www.ctan.org/pkg/cite
% The documentation is contained in the cite.sty file itself.






% *** GRAPHICS RELATED PACKAGES ***
%
\ifCLASSINFOpdf
  % \usepackage[pdftex]{graphicx}
  % declare the path(s) where your graphic files are
  % \graphicspath{{../pdf/}{../jpeg/}}
  % and their extensions so you won't have to specify these with
  % every instance of \includegraphics
  % \DeclareGraphicsExtensions{.pdf,.jpeg,.png}
\else
  % or other class option (dvipsone, dvipdf, if not using dvips). graphicx
  % will default to the driver specified in the system graphics.cfg if no
  % driver is specified.
  % \usepackage[dvips]{graphicx}
  % declare the path(s) where your graphic files are
  % \graphicspath{{../eps/}}
  % and their extensions so you won't have to specify these with
  % every instance of \includegraphics
  % \DeclareGraphicsExtensions{.eps}
\fi
% graphicx was written by David Carlisle and Sebastian Rahtz. It is
% required if you want graphics, photos, etc. graphicx.sty is already
% installed on most LaTeX systems. The latest version and documentation
% can be obtained at: 
% http://www.ctan.org/pkg/graphicx
% Another good source of documentation is "Using Imported Graphics in
% LaTeX2e" by Keith Reckdahl which can be found at:
% http://www.ctan.org/pkg/epslatex
%
% latex, and pdflatex in dvi mode, support graphics in encapsulated
% postscript (.eps) format. pdflatex in pdf mode supports graphics
% in .pdf, .jpeg, .png and .mps (metapost) formats. Users should ensure
% that all non-photo figures use a vector format (.eps, .pdf, .mps) and
% not a bitmapped formats (.jpeg, .png). The IEEE frowns on bitmapped formats
% which can result in "jaggedy"/blurry rendering of lines and letters as
% well as large increases in file sizes.
%
% You can find documentation about the pdfTeX application at:
% http://www.tug.org/applications/pdftex





% *** MATH PACKAGES ***
%
%\usepackage{amsmath}
% A popular package from the American Mathematical Society that provides
% many useful and powerful commands for dealing with mathematics.
%
% Note that the amsmath package sets \interdisplaylinepenalty to 10000
% thus preventing page breaks from occurring within multiline equations. Use:
%\interdisplaylinepenalty=2500
% after loading amsmath to restore such page breaks as IEEEtran.cls normally
% does. amsmath.sty is already installed on most LaTeX systems. The latest
% version and documentation can be obtained at:
% http://www.ctan.org/pkg/amsmath





% *** SPECIALIZED LIST PACKAGES ***
%
%\usepackage{algorithmic}
% algorithmic.sty was written by Peter Williams and Rogerio Brito.
% This package provides an algorithmic environment fo describing algorithms.
% You can use the algorithmic environment in-text or within a figure
% environment to provide for a floating algorithm. Do NOT use the algorithm
% floating environment provided by algorithm.sty (by the same authors) or
% algorithm2e.sty (by Christophe Fiorio) as the IEEE does not use dedicated
% algorithm float types and packages that provide these will not provide
% correct IEEE style captions. The latest version and documentation of
% algorithmic.sty can be obtained at:
% http://www.ctan.org/pkg/algorithms
% Also of interest may be the (relatively newer and more customizable)
% algorithmicx.sty package by Szasz Janos:
% http://www.ctan.org/pkg/algorithmicx




% *** ALIGNMENT PACKAGES ***
%
%\usepackage{array}
% Frank Mittelbach's and David Carlisle's array.sty patches and improves
% the standard LaTeX2e array and tabular environments to provide better
% appearance and additional user controls. As the default LaTeX2e table
% generation code is lacking to the point of almost being broken with
% respect to the quality of the end results, all users are strongly
% advised to use an enhanced (at the very least that provided by array.sty)
% set of table tools. array.sty is already installed on most systems. The
% latest version and documentation can be obtained at:
% http://www.ctan.org/pkg/array


% IEEEtran contains the IEEEeqnarray family of commands that can be used to
% generate multiline equations as well as matrices, tables, etc., of high
% quality.




% *** SUBFIGURE PACKAGES ***
%\ifCLASSOPTIONcompsoc
%  \usepackage[caption=false,font=normalsize,labelfont=sf,textfont=sf]{subfig}
%\else
%  \usepackage[caption=false,font=footnotesize]{subfig}
%\fi
% subfig.sty, written by Steven Douglas Cochran, is the modern replacement
% for subfigure.sty, the latter of which is no longer maintained and is
% incompatible with some LaTeX packages including fixltx2e. However,
% subfig.sty requires and automatically loads Axel Sommerfeldt's caption.sty
% which will override IEEEtran.cls' handling of captions and this will result
% in non-IEEE style figure/table captions. To prevent this problem, be sure
% and invoke subfig.sty's "caption=false" package option (available since
% subfig.sty version 1.3, 2005/06/28) as this is will preserve IEEEtran.cls
% handling of captions.
% Note that the Computer Society format requires a larger sans serif font
% than the serif footnote size font used in traditional IEEE formatting
% and thus the need to invoke different subfig.sty package options depending
% on whether compsoc mode has been enabled.
%
% The latest version and documentation of subfig.sty can be obtained at:
% http://www.ctan.org/pkg/subfig




% *** FLOAT PACKAGES ***
%
%\usepackage{fixltx2e}
% fixltx2e, the successor to the earlier fix2col.sty, was written by
% Frank Mittelbach and David Carlisle. This package corrects a few problems
% in the LaTeX2e kernel, the most notable of which is that in current
% LaTeX2e releases, the ordering of single and double column floats is not
% guaranteed to be preserved. Thus, an unpatched LaTeX2e can allow a
% single column figure to be placed prior to an earlier double column
% figure.
% Be aware that LaTeX2e kernels dated 2015 and later have fixltx2e.sty's
% corrections already built into the system in which case a warning will
% be issued if an attempt is made to load fixltx2e.sty as it is no longer
% needed.
% The latest version and documentation can be found at:
% http://www.ctan.org/pkg/fixltx2e


%\usepackage{stfloats}
% stfloats.sty was written by Sigitas Tolusis. This package gives LaTeX2e
% the ability to do double column floats at the bottom of the page as well
% as the top. (e.g., "\begin{figure*}[!b]" is not normally possible in
% LaTeX2e). It also provides a command:
%\fnbelowfloat
% to enable the placement of footnotes below bottom floats (the standard
% LaTeX2e kernel puts them above bottom floats). This is an invasive package
% which rewrites many portions of the LaTeX2e float routines. It may not work
% with other packages that modify the LaTeX2e float routines. The latest
% version and documentation can be obtained at:
% http://www.ctan.org/pkg/stfloats
% Do not use the stfloats baselinefloat ability as the IEEE does not allow
% \baselineskip to stretch. Authors submitting work to the IEEE should note
% that the IEEE rarely uses double column equations and that authors should try
% to avoid such use. Do not be tempted to use the cuted.sty or midfloat.sty
% packages (also by Sigitas Tolusis) as the IEEE does not format its papers in
% such ways.
% Do not attempt to use stfloats with fixltx2e as they are incompatible.
% Instead, use Morten Hogholm'a dblfloatfix which combines the features
% of both fixltx2e and stfloats:
%
% \usepackage{dblfloatfix}
% The latest version can be found at:
% http://www.ctan.org/pkg/dblfloatfix




%\ifCLASSOPTIONcaptionsoff
%  \usepackage[nomarkers]{endfloat}
% \let\MYoriglatexcaption\caption
% \renewcommand{\caption}[2][\relax]{\MYoriglatexcaption[#2]{#2}}
%\fi
% endfloat.sty was written by James Darrell McCauley, Jeff Goldberg and 
% Axel Sommerfeldt. This package may be useful when used in conjunction with 
% IEEEtran.cls'  captionsoff option. Some IEEE journals/societies require that
% submissions have lists of figures/tables at the end of the paper and that
% figures/tables without any captions are placed on a page by themselves at
% the end of the document. If needed, the draftcls IEEEtran class option or
% \CLASSINPUTbaselinestretch interface can be used to increase the line
% spacing as well. Be sure and use the nomarkers option of endfloat to
% prevent endfloat from "marking" where the figures would have been placed
% in the text. The two hack lines of code above are a slight modification of
% that suggested by in the endfloat docs (section 8.4.1) to ensure that
% the full captions always appear in the list of figures/tables - even if
% the user used the short optional argument of \caption[]{}.
% IEEE papers do not typically make use of \caption[]'s optional argument,
% so this should not be an issue. A similar trick can be used to disable
% captions of packages such as subfig.sty that lack options to turn off
% the subcaptions:
% For subfig.sty:
% \let\MYorigsubfloat\subfloat
% \renewcommand{\subfloat}[2][\relax]{\MYorigsubfloat[]{#2}}
% However, the above trick will not work if both optional arguments of
% the \subfloat command are used. Furthermore, there needs to be a
% description of each subfigure *somewhere* and endfloat does not add
% subfigure captions to its list of figures. Thus, the best approach is to
% avoid the use of subfigure captions (many IEEE journals avoid them anyway)
% and instead reference/explain all the subfigures within the main caption.
% The latest version of endfloat.sty and its documentation can obtained at:
% http://www.ctan.org/pkg/endfloat
%
% The IEEEtran \ifCLASSOPTIONcaptionsoff conditional can also be used
% later in the document, say, to conditionally put the References on a 
% page by themselves.




% *** PDF, URL AND HYPERLINK PACKAGES ***
%
%\usepackage{url}
% url.sty was written by Donald Arseneau. It provides better support for
% handling and breaking URLs. url.sty is already installed on most LaTeX
% systems. The latest version and documentation can be obtained at:
% http://www.ctan.org/pkg/url
% Basically, \url{my_url_here}.




% *** Do not adjust lengths that control margins, column widths, etc. ***
% *** Do not use packages that alter fonts (such as pslatex).         ***
% There should be no need to do such things with IEEEtran.cls V1.6 and later.
% (Unless specifically asked to do so by the journal or conference you plan
% to submit to, of course. )


% correct bad hyphenation here
\hyphenation{op-tical net-works semi-conduc-tor}


\begin{document}
%
% paper title
% Titles are generally capitalized except for words such as a, an, and, as,
% at, but, by, for, in, nor, of, on, or, the, to and up, which are usually
% not capitalized unless they are the first or last word of the title.
% Linebreaks \\ can be used within to get better formatting as desired.
% Do not put math or special symbols in the title.
\title{The Structure  of Decoherence-free Subsystems}
%
%
% author names and IEEE memberships
% note positions of commas and nonbreaking spaces ( ~ ) LaTeX will not break
% a structure at a ~ so this keeps an author's name from being broken across
% two lines.
% use \thanks{} to gain access to the first footnote area
% a separate \thanks must be used for each paragraph as LaTeX2e's \thanks
% was not built to handle multiple paragraphs
%

\author{Ji~Guan,%~\IEEEmembership{Member,~IEEE,}
        ~Yuan~Feng%~\IEEEmembership{Fellow,~OSA,}
        ~and~Mingsheng~Ying%~\IEEEmembership{Life~Fellow,~IEEE}% <-this % stops a space
\thanks{Ji Guan and Yuan Feng are with Centre for Quantum Software and Information,
University of Technology Sydney, NSW 2007, Australia.}% <-this % stops a space
\thanks{Mingsheng Ying is with Centre for Quantum Software and Information,
University of Technology Sydney, NSW 2007, Australia and Department of Computer Science and Technology, Tsinghua University, Beijing 100084, China.}}% <-this % sto

% note the % following the last \IEEEmembership and also \thanks - 
% these prevent an unwanted space from occurring between the last author name
% and the end of the author line. i.e., if you had this:
% 
% \author{....lastname \thanks{...} \thanks{...} }
%                     ^------------^------------^----Do not want these spaces!
%
% a space would be appended to the last name and could cause every name on that
% line to be shifted left slightly. This is one of those "LaTeX things". For
% instance, "\textbf{A} \textbf{B}" will typeset as "A B" not "AB". To get
% "AB" then you have to do: "\textbf{A}\textbf{B}"
% \thanks is no different in this regard, so shield the last } of each \thanks
% that ends a line with a % and do not let a space in before the next \thanks.
% Spaces after \IEEEmembership other than the last one are OK (and needed) as
% you are supposed to have spaces between the names. For what it is worth,
% this is a minor point as most people would not even notice if the said evil
% space somehow managed to creep in.



% The paper headers
\markboth{Journal of \LaTeX\ Class Files,~Vol.~14, No.~8, August~2015}%
{Shell \MakeLowercase{\textit{et al.}}: Bare Demo of IEEEtran.cls for IEEE Journals}
% The only time the second header will appear is for the odd numbered pages
% after the title page when using the twoside option.
% 
% *** Note that you probably will NOT want to include the author's ***
% *** name in the headers of peer review papers.                   ***
% You can use \ifCLASSOPTIONpeerreview for conditional compilation here if
% you desire.




% If you want to put a publisher's ID mark on the page you can do it like
% this:
%\IEEEpubid{0000--0000/00\$00.00~\copyright~2015 IEEE}
% Remember, if you use this you must call \IEEEpubidadjcol in the second
% column for its text to clear the IEEEpubid mark.



% use for special paper notices
%\IEEEspecialpapernotice{(Invited Paper)}




% make the title area
\maketitle

% As a general rule, do not put math, special symbols or citations
% in the abstract or keywords.
\begin{abstract}
Decoherence-free subsystems have been successfully developed as a tool to preserve fragile quantum information against noises.  In this work, we develop a structure theory for decoherence-free subsystems. Consequently, under perfect initialization of quantum states,  we propose an effective algorithm to find an optimal set of  decoherence-free subsystems for any given quantum super-operator such that any other such subsystem is a subspace of one of them. Furthermore, we give an easy way to pick up faulty-tolerant ones from them. Using these mathematical techniques, we obtain a simple numerical  method to obtain a basis for any tensor, generating  a family of matrix product states.
\end{abstract}

% Note that keywords are not normally used for peerreview papers.
\begin{IEEEkeywords}
decoherence-free subsystems, continuous coherences, faulty-tolerance, matrix product states, structure theory.
\end{IEEEkeywords}






% For peer review papers, you can put extra information on the cover
% page as needed:
% \ifCLASSOPTIONpeerreview
% \begin{center} \bfseries EDICS Category: 3-BBND \end{center}
% \fi
%
% For peerreview papers, this IEEEtran command inserts a page break and
% creates the second title. It will be ignored for other modes.
\IEEEpeerreviewmaketitle



\section{Introduction}
% The very first letter is a 2 line initial drop letter followed
% by the rest of the first word in caps.
% 
% form to use if the first word consists of a single letter:
% \IEEEPARstart{A}{demo} file is ....
% 
% form to use if you need the single drop letter followed by
% normal text (unknown if ever used by the IEEE):
% \IEEEPARstart{A}{}demo file is ....
% 
% Some journals put the first two words in caps:
% \IEEEPARstart{T}{his demo} file is ....
% 
% Here we have the typical use of a "T" for an initial drop letter
% and "HIS" in caps to complete the first word.
\IEEEPARstart{T}{o} 
% You must have at least 2 lines in the paragraph with the drop letter
% (should never be an issue)
build large scale quantum computers, the obstacles, such as decoherences and noises,  must be managed and overcome~\cite{nielsen2010quantum}. One of the effective methods for this purpose is through
decoherence-free subspaces proposed by Daniel A. Lidar in \cite{lidar1998decoherence}. A subspace of the system Hilbert space is said to be decoherence-free if the effect of the noise on it is simply unitary, and thus easily correctable. For this sake, decoherence-free subspaces are important subjects in quantum computing, where coherent control of quantum systems is often the desired goal \cite{lidar2012review}. On the other hand, decoherence-free subspaces  can also be characterized as a special case of quantum error correcting codes to preserve quantum information against noises \cite{lidar2012review}. However, we do not need to restrict the decoherence-free dynamics to a subspace. E. Knill, R. Laflamme, and L. Viola introduced  the concept of noiseless subsystems, by extending higher-dimensional irreducible representations of the algebra generating the dynamical symmetry in the system-environment interaction \cite{knill2000theory}. A subsystem is a factor in a tensor product decomposition of a subspace and 
the noiseless subsystem requires the evolution on it to be strict identity. Such noiseless subsystems have been fully characterized and intensely studied in \cite{choi2006method,blume2010information,beny2007generalization,kribs2006quantum,kribs2005unified}. Remarkably, the structure theory of noiseless subsystems  was proposed in \cite{choi2006method}, leading to an algorithm which finds all noiseless subsystems for a given quantum super-operator \cite{knill2006protected,wang2013numerical}. In the meanwhile, the general case of decoherence-free subspaces and noiseless subsystems, called decoherence-free subsystems, was examined and the conditions for their existence were found in \cite{shabani2005theory}, and subsystems with significantly reduced noises  were considered in \cite{wang2016minimal}. However, a clear picture of the structure of  decoherence-free subsystems (subspaces) is still lacking, and it is hard to compute all decoherence-free subsystems (subspaces) or the highest-dimensional ones for any given quantum super-operator. 

In this paper,  we develop a structure theory that shows precisely how a super-operator, the evolution of a quantum system, determines its decoherence-free subsystems, generating the existing results for noiseless subsystems. As an application, under perfect initialization of quantum states, we develop an algorithm (Algorithm 1) to generate an optimal set of decoherence-free subsystems for any given super-operator such that any other decoherence-free subsystem is a subspace of one of them. We then consider decoherence-free subsystems with imperfect initialization, and obtain a very simple  method to find them. 
We further use this mathematical tool in the quantum many-body system described by a family of matrix product states generated by a tensor and propose a  feasible way to numerically derive   a basis for the tensor. The basis plays an important role in fundamental theorems of Matrix product states \cite{cirac2017matrix,cuevas2017irreducible}.

This paper is organized as follows. We recall some basic notions of quantum information theory and introduce one central concept, continuous coherences in Section \uppercase\expandafter{\romannumeral2}. In Section \uppercase\expandafter{\romannumeral3}, we review the structure theory of noiseless subsystems by studying the fixed points of super-operators. We then show a similar structure theory of decoherence-free subsystems in Section \uppercase\expandafter{\romannumeral4}, which leads to an algorithm of constructing an optimal set of  such subsystems with the assumption perfect initialization, for a given super-operator. In Section \uppercase\expandafter{\romannumeral5}, we present a procedure for checking whether or not a subsystem with a co-subsystem is decoherence-free under imperfect initialization. Furthermore, 
we apply previous results to find a basis for any tensor, generating a set of matrix product states to represent a quantum many-body system  in Section \uppercase\expandafter{\romannumeral6}. A brief conclusion is drawn in the last section.

\section{Preliminaries}

In this section, for convenience of the reader, we review some basic notions and results from quantum information theory; for details we refer to \cite{nielsen2010quantum}. Recall that given a quantum system $S$ with the associated (finite-dimensional) state Hilbert space $\h$, the evolution of the system can be mathematically  modeled by a super-operator, i.e., a completely positive  and trace-preserving (CPTP) map $\e$ on $\h$.  We say that a quantum system $A$ is a subsystem of $S$ if $\h=(\h_A\bigotimes\h_B)\bigoplus \k$ for some co-subsystem $B$, where $\k=(\h_A\bigotimes\h_B)^\perp$, $\h_A$ and $\h_B$ are the state spaces  of $A$ and $B$, respectively.  
  For any two Hilbert spaces $\h$ and $\h'$, let $\l(\h,\h')$ be the set of all  operators from $\h$ to $\h'$. Simply, we define that $\l(\h)=\l(\h,\h)$ and $\d(\h)$ is the set of all quantum states, i.e. density operators with unit trace, on $\h$.  The support of a quantum state $\rho$, denoted by supp$(\rho)$, is the linear span of the eigenvectors corresponding to non-zero eigenvalues of $\rho.$ 
\begin{definition}
  Given a super-operator $\e$ on $\h$, 
  \begin{itemize}
    \item[(1)] a quantum state  $\rho$ is said to be stationary if it is a fixed point of $\e,$ i.e. $\e(\rho)=\rho.$ Furthermore, $\rho$ is minimal if there is no other stationary state $\sigma$ such that supp$(\sigma)\subseteq \textrm{supp}(\rho);$
    \item[(2)] a subspace $\h_1\subseteq \h$ is minimal if it is a support of some minimal stationary state. Furthermore, if the whole space $\h$ is minimal, then we call it irreducible; otherwise it is reducible.
    \item[(3)] a subspace $\k$ is called transient if for any state $\rho\in \d(\h)$, $$\lim_{n\rightarrow \infty}\textrm{tr}(P_\k\e^n(\rho))=0,$$ where $P_\k$ is the projector onto $\k$. 
  \end{itemize} 
\end{definition}

Note that if two minimal states $\rho$ and $\sigma$ have the same support, then $\rho=\sigma$. Thus $\h$ is irreducible if and only if there is only one stationary state.


Applying the techniques developed in \cite{ying2013reachability,baumgartner2012structure}, we can decompose $\h$ into mutually orthogonal minimal subspaces and the largest transient subspace: 
\begin{eqnarray}\label{eq_mini_dec}
  \h=\bigoplus_{p=1}^m\h_p\bigoplus\k.
\end{eqnarray}
Furthermore, the Kraus operators $\{E_k\}$ of $\e$ have the  corresponding block form:
$$\renewcommand{\arraystretch}{1.2}
E_k=\left[\begin{array}{c|c}
  \begin{array}{cccc}
  E_{k,1} & & &\\
  & E_{k,2}  & &\\
  & & \ddots &\\
  & & & E_{k,m}  
  \end{array} & T_k\\
  \hline
0&K_k
\end{array}\right]$$
 for some operators $E_{k,p}\in \l(\h_p)$, $K_k\in \l(\k)$, and $T_k\in \l(\k,\k^{\perp}).$
We then define a set of associated maps $\{\e_{p,q} : p,q = 1, \dots, m\}$ of $\e$:
\begin{eqnarray}\label{Eq_ass_maps}
  \e_{p,q}(\cdot)=\sum_{k} E_{k,p}\cdot E_{k,q}^\dagger.
\end{eqnarray}
Obviously, for any $p$ and $q$, $\e_{p,q}$ is a linear  map from $\l(\h_q,\h_p)$ to itself. If $p\not =q$, $\l(\h_q, \h_p)$ can  be viewed as (outer) coherences from  $\h_q$ to $\h_p$, i.e.  upper off-diagonal blocks of all matrices restricted in the decomposition $\h_p\bigoplus\h_q.$ Thus the coherence between $\h_p$ and $\h_q$ is  $\l(\h_q, \h_p)\bigoplus\l(\h_p, \h_q)$ and $\l(\h_q)$ can be regarded as inner coherences.

For all $p$ and $q$, the following two properties are easy to observe:
\begin{enumerate}
\item $\l(\h_p,\h_q)$ is invariant under $\e$; that is, for all $A\in \l(\h_p,\h_q)$, $\e(A)\in \l(\h_p,\h_q)$.
\item  $\lambda(\e_{p,q})\subseteq \lambda(\e)$, where $\lambda(\cdot)$ is the set of eigenvalues of a linear map.
\end{enumerate}
Furthermore,  the coherence $\l(\h_p,\h_q)$ is said to be continuous if there exists $A\in \l(\h_p,\h_q)$ such that $\e(A)=e^{i\theta}A$ for some real number $\theta$; that is, $\lambda(\e_{p,q})$ has an element with magnitude one.  Specially, if $\theta=0$, then $\l(\h_p,\h_q)$ is stationary. Obviously, $\l(\h_p,\h_p)$ is always stationary because a super-operator has at least one stationary state. Stationary coherences have been intensely studied in~\cite{baumgartner2012structure}, where a nice structure of $fix(\e)$, the set of fixed points of super-operator $\e$, is discovered. We will restate this result in the next section.

\begin{definition}\label{eq_dfs_def}
   Let $\e$ be a super-operator on $\h=\h_A\bigotimes\h_B\bigoplus\k $.
   $\h_A$ is a decoherence-free subsystem if there is a unitary matrix $U_A$ on $\h_A$ such that for any initial state $\rho$,\begin{eqnarray}\label{Eq_def_U}
     \textrm{tr}_{B}[\e(P_{AB}\rho P_{AB})]=U_A\textrm{tr}_{B}[(P_{AB}\rho P_{AB})]U_A^\dagger,
   \end{eqnarray}
   where $P_{AB}$ is the projector onto $\h_A\bigotimes \h_B$. 
  Furthermore, if $U_A=I_A$, the identity operator on $\h_A$, then we say that $\h_A$ is noiseless.  
\end{definition}


%In an open quantum system $\h$ which is under the dynamic $\e$, decoherence-free subsystem $\h_A$ is a closed quantum system, i.e. the evolution on it  is completely unitary. Thus quantum information can be stored faithfully in $\h_A$.

Given a decomposition of $\h=\h_A\bigotimes \h_B\bigoplus\k$ and a initial state $\rho\in D(\h)$, $\rho$ has the following block form:
  \begin{eqnarray}
  \rho=\left[ \begin{matrix}
  \rho_{AB}&\rho'\\
  \rho'^\dagger&\rho_\k
  \end{matrix}\right]
\end{eqnarray}
If $\h_A$ is decoherence-free, then the quantum information $\textrm{tr}_{B}(\rho_{AB})$ can be preserved in it, even though $\textrm{tr}_{B}(\rho_{AB})$ is not a valid quantum state as its  trace might be less than one. Ideally, we hope that $\rho$ is initialized in $\h_A\bigotimes\h_B$, i.e. $\rho'=0$ and $\rho_\k=0$; this situation is called perfect initialization. 
  But in many cases, specially experiments,  such initialization might be challenging, so we will face imperfect initialization, i.e. $\rho'\not =0$ or $\rho_\k\not =0$.
 
  The results of \cite{shabani2005theory} have shown that in perfect initialization, $\h_A$ is decoherence-free if and only if the Kraus operators $\{E_k\}_k$ of $\e$ have the matrix representation:
\begin{eqnarray}\label{Eq_perfect}
  E_{k}=\left [\begin{matrix}
  U\otimes E'_{k}&T_k\\
  0&K_k
\end{matrix}\right]\ \ \forall k
\end{eqnarray}
; in imperfect initialization, we have that
\begin{eqnarray}\label{Eq_imperfect}
  E_{k}=\left [\begin{matrix}
  U\otimes E'_{k}&0\\
  0&K_k
\end{matrix}\right]\ \ \forall k
\end{eqnarray}

Obviously, co-subsystem $\h_B$ is important in later case, but is inessential in former case as the initial state is prepared in $\h_A\bigotimes \h_B$ and $\h_B$ can be traced over. A decoherence-free subsystem $\h_A$ under imperfect initialization is always decoherence-free under perfect initialization, but the converse part is not true in all cases. 

In the following two sections, we assume perfect initialization, i.e. initial states can be prepared exactly in any subspace. In Section \uppercase\expandafter{\romannumeral5},  we will deal with imperfect initialization.  

\section{Fixed Points and Noiseless Subsystems}
Noiseless subsystems are a special case of decoherence-free subsystems and have been intensely studied in quantum error correction \cite{kribs2005operator,beny2007generalization} and quantum memory \cite{kuperberg2003capacity}. As we are going to show a structure theory of decoherence-free subsystems in the upcoming section, we first  review the counterpart in noiseless subsystems which is inspired by the structure of fixed points of super-operators.  

To characterize  $fix(\e)$, the main step is to identify stationary coherences of minimal subspaces in  the decomposition of Eq.(\ref{eq_mini_dec}) and this can be achieved  by the following lemma.

\begin{lemma}[\cite{baumgartner2012structure}]\label{Lem_SC}
  Let $\e$ be a super-operator on $\h$ with the orthogonal decomposition presented in Eq.(\ref{eq_mini_dec}). For any $1\leq p, q\leq m$, $\l(\h_p,\h_q)$ is stationary if and only if there is a unitary matrix $U$ such that 
  $E_{k,p}=UE_{k,q}U^\dagger$ for all $k$.
  %, where $\{E_{k,p}\}_k$ and $\{E_{k,q}\}_k$ are the restriction of  Kraus operators of $\e$ onto $\h_p$ and $\h_q$, respectively.  
  Furthermore, $\h_p\simeq\h_q.$
\end{lemma}

 From Lemma~\ref{Lem_SC}, $\l(\h_p,\h_q)$ is stationary if and only if so is $\l(\h_q,\h_p)$. Thus in the following, we simply say that there is a stationary coherence between $\h_p$ and $\h_q$ without referring to the direction.
Furthermore,  we group together minimal subspaces by stationary coherences and obtain a structure of $fix(\e)$ as follows. 
\begin{theorem}[\cite{baumgartner2012structure}]\label{Theo_NS_Dec}
  Let $\e$ be a super-operator on $\h$. There is a unique  orthogonal decomposition of $\h$
  \begin{eqnarray}\label{Eq_NS_dec_unique}
    \h=\bigoplus_{l=1}^{L}\x_l\bigoplus \k
  \end{eqnarray}
  where \begin{itemize}
    \item[(1)] $\k$ is the largest transient subspace;
    \item[(2)] each $\x_l$ is either a minimal subspace or can be further decomposed into mutually orthogonal minimal subspaces with stationary coherences between any two of them:
    \begin{eqnarray}\label{Eq_NS_dec}
      \x_l=\bigoplus_{p=1}^{m_l}\b_{l,p}\simeq\mathbb{C}^{m_l}\bigotimes \b_l, \ \ \b_l\simeq \b_{l,p} \ \forall p
    \end{eqnarray} 
    so that the Kraus operators $\{E_k\}$ of $\e$ have a unitarily equivalent block form:
  \begin{eqnarray}\label{Eq_SC_dec_Kraus}
    \renewcommand{\arraystretch}{1.2}
E_k\simeq \left[\begin{array}{c|c}
  \begin{array}{ccc}
  I_{1}\otimes E_{k,1} &  &\\
   & \ddots &\\
   & & I_{L}\otimes E_{k,L}  
  \end{array} & T_k\\
  \hline
0&K_k
\end{array}\right]
  \end{eqnarray}

for some operators $E_{k,l}\in \l(\b_l)$, $K_k\in \l(\k)$, and $T_k\in\l(\k,\k^\perp)$. Here  $I_{l}$ is the identity operator on $\mathbb{C}^{m_l}$ and $\b_l$ is irreducible under $\e_{l}(\cdot)=\sum_{k}E_{k,l}\cdot E_{k,l}^\dagger$. Furthermore, 
$$fix(\e)\simeq\bigoplus_{l}[\l(\mathbb{C}^{m_l})\otimes \rho_{l}]\bigoplus 0_\k $$
    where $\rho_l$ is the unique stationary state of $\e_{l}$, and $0_\k$ is the zero operator on $\k$.
    \item[(3)] there is no stationary coherence between any minimal subspaces $\b_{l, p}$ and $\b_{l', p'}$ whenever $l\neq l'.$  
      \end{itemize}
\end{theorem}

In the following parts of this paper we will, with slight abuse of formulation,
write all the formulas related to splitting $\h$ as in Eq.(\ref{Eq_NS_dec})  or Kraus operators as in Eq.(\ref{Eq_SC_dec_Kraus}) with ``$=$'' instead of ``$\simeq$''.

Actually, as the following theorem shows, all noiseless subsystems have been captured by the above decomposition.
\begin{theorem}[\cite{blume2010information}]
  Let $\e$ be a super-operator on $\h$ with its unique orthogonal decomposition
   $$\h=\bigoplus_{l=1}^m\left[\mathbb{C}^{m_l}\bigotimes\b_l\right]\bigoplus\k,$$
   presented in Theorem~\ref{Theo_NS_Dec}. Then
    a subsystem $\h_A$ is noiseless if and only if $\h_A\subseteq \mathbb{C}^{m_l}$ for some $l$.
\end{theorem}

The decomposition in Theorem~\ref{Theo_NS_Dec} can also be obtained by applying the structure of  $C^*$-algebra generated by the Kraus operators of $\e$; see \cite{choi2006method} for details. Subsequently, some algorithms for implementing the above decompositions in Eqs.~(\ref{Eq_NS_dec_unique}-\ref{Eq_SC_dec_Kraus}) were developed from the structure of the $C^*$-algebra   or $fix(\e)$   \cite{guan2016decomposition,knill2006protected,wang2013numerical}.  Such algorithms will be referred as NSDecompose($\h,\e$) in this paper and the time complexity is $O(n^{8})$, where dim$(\h)=n$. 

\begin{example}
  Given $\h=\h_A\bigotimes \h_B$, and $\{|k\rangle_A\}_{k=0}^3$ and $\{|k\rangle_B\}_{k=0}^2$ are orthonormal bases of $\h_A$ and $\h_B$, respectively, let $\e$ be a super-operator on $\h$ with the Kraus operators:
  \begin{eqnarray*}
    E_{1}&=&|00\rangle\langle01|+|10\rangle\langle11|-|20\rangle\langle21|-|30\rangle\langle31|\\
    E_{2}&=&|01\rangle\langle00|+|11\rangle\langle10|-|21\rangle\langle20|-|31\rangle\langle30|\\
    E_{3}&=&|00\rangle\langle02|+|10\rangle\langle12|-|20\rangle\langle22|-|30\rangle\langle32|
  \end{eqnarray*}
  where $|kl\rangle=|k\rangle_A\otimes |l\rangle_B$. It is easy to calculate the unique decomposition of $\h$ in Theorem~\ref{Theo_NS_Dec} as 
  $$\h=\bigoplus_{l=1}^2\left[\h_l\bigotimes \h'\right] \bigoplus\k$$
  where $\h_1=\textrm{lin.span}\{|0\rangle_A,|1\rangle_A\}$, $\h_2=\textrm{lin.span}\{|2\rangle_A,|3\rangle_A\}$, $\h'=\textrm{lin.span}\{|0\rangle_B,|1\rangle_B\}$, and $\k = \h_A\bigotimes \textrm{lin.span}\{|2\rangle_B\}$. Then we can store 2-qubit quantum information in $\h_1$ or $\h_2$.
\end{example}

\section{Decoherence-free Subsystems with Perfect Initialization}
In this section, a similar orthogonal decomposition as that in Theorem~\ref{Theo_NS_Dec} is proposed for decoherence-free subsystems. Employing it, we then develop an efficient algorithm to find all such subsystems for any given super-operator. 

By the definition, a decoherence-free subsystem $\h_A$  is a small place of Hilbert space $\h$ with a unitary evolution under the quantum noise, modeled by a super-operator $\e$. From Eq.(\ref{Eq_perfect}), the restriction of $\e$ onto $\h_A\bigotimes\h_B $, where $\h_B$ is the co-subsystem of $\h_A$, can be written as \begin{eqnarray}\label{eq_dfs_form}
  \e_{AB}=\u_A\otimes \e_{B}
\end{eqnarray} where $\u_A$ is a unitary super-operator on $\h_A$ and $\e_B$ is a super-operator on $\h_{B}.$  By the decomposition Eq.(\ref{eq_mini_dec}), $\h_B$ can be chosen to be  irreducible. In this section, we assume that the co-subsystem of a decoherence-free subsystem is always irreducible. 

First, we observe that  the joint systems of decoherence-free subsystems and irreducible co-subsystems consist of minimal subspaces with continuous coherences. 
\begin{theorem}\label{Theo_dfs_cc}
    Given a super-operator $\e$ on 
    $$\h=\h_A\bigotimes\h_B\bigoplus \k,$$ let $\h_A$ is a decoherence-free subsystem and $U_A$ the corresponding unitary matrix in Eq.(\ref{Eq_def_U}). Let  $\{|p\rangle\}_{p=1}^{m}$ be a set of mutually orthogonal eigenvectors of $U_A$ and $\h_p=\textrm{lin.span}\{|p\rangle\}\bigotimes \h_B$. Then for all $1\leq p,q\leq m$,  $\h_p$  is a minimal subspace and
    $\l(\h_p,\h_q)$ is continuous.
  \end{theorem}  

{\it Proof.} Note that we assume $\h_B$ is irreducible. Let $\rho$ be the unique stationary state of $\e_B$.  Then for any $p$, $|p\rangle\langle p|\otimes \rho$ is a minimal stationary state of $\e$, and hence $\h_p$ is minimal. Furthermore, note that $U_A|p\rangle=e^{i\theta_p}|p\rangle$ for some $\theta_p$. Thus $\e(|p\rangle\langle q|\otimes \rho)=e^{i(\theta_p-\theta_q)}|p\rangle\langle q|\otimes \rho$ for all $p$ and $q.$
\hfill $\Box$

Theorem~\ref{Theo_dfs_cc} indicates that minimal subspaces with continuous coherences play an important role in determining decoherence-free subsystems. To check if two orthogonal minimal subspaces have continuous coherence, we present the following lemma which is similar to Lemma~\ref{Lem_SC} for stationary coherences.

\begin{lemma}\label{lem_block_eq}
  Let $\e$ be a super-operator on $\h$ with the orthogonal decomposition presented in Eq.(\ref{eq_mini_dec}). For any $1\leq p, q\leq m$, $\l(\h_p,\h_q)$ is continuous if and only if there is a unitary matrix $U$ and a real number $\theta$ such that 
  $E_{k,p}=e^{i\theta}UE_{k,q}U^\dagger$ for all $k$.
  %, where $\{E_{k,p}\}_k$ and $\{E_{k,q}\}_k$ are the restriction of  Kraus operators of $\e$ onto $\h_p$ and $\h_q$, respectively.  
  Furthermore, $\h_p\simeq\h_q.$ 
\end{lemma}
{\it Proof.} 
Assume that $\l(\h_p,\h_q)$ is continuous; that is,  there is a matrix $A\in \l(\h_p,\h_q)$ such that $\e(A)=e^{i\theta}A$ for some real number $\theta$. Let $V=e^{-i\theta}P_q+I-P_q$, where $P_p$ is the projector onto $\h_p$,  and  $\v\circ\e(A)=A$ with $\v(\cdot)=V\cdot V^\dagger$. Moreover, it is obvious that $\h_p$ and $\h_q$ are also orthogonal minimal subspaces under $\v\circ\e$  by the decomposition Eq.(\ref{eq_mini_dec}). Therefore, there is a stationary coherence from $\h_q$ to $\h_p$ under $\v\circ\e.$ From Lemma~\ref{Lem_SC} and the minimal decomposition of $\v\circ\e$, we have $\h_p\simeq\h_q$, and there exists some unitary matrix $U$ such that that for any $k$,
$$E_{k,p}=e^{i\theta}UE_{k,q}U^\dagger.$$ 

Conversely, for any $p$ and $q$ let $$\e_{p,q}(\cdot)=\sum_{k}E_{k,p}\cdot E_{k,q}^\dagger=\sum_{k}e^{i\theta}UE_{k,q}U^\dagger\cdot E_{k,q}^\dagger.$$ 
Its matrix representation \cite{guan2016decomposition} reads
\begin{eqnarray*}
  M_{p,q}&=&\sum_k e^{i\theta}UE_{k,q}U^\dagger\otimes E_{k,q}^{*}\\
  &=&e^{i\theta}(U\otimes I)\left(\sum_k E_{k,q}\otimes E_{k,q}^{*}\right)(U^\dagger\otimes I)
\end{eqnarray*}
As $\lambda(M_{p,q})=\lambda(\e_{p,q})$ and $\sum_k E_{k,q}\otimes E_{k,q}^{*}$ is the matrix presentation of $\e_{q,q}$ which is a super-operator and has 1 as one of its eigenvalues, we have $e^{i\theta}\in\lambda(\e_{p,q}).$  
\hfill $\Box$

%The above lemma is a generalization of Lemma~\ref{lem_block_eq} in which stationary coherences need the Kraus operators are one-by-one unitarily equivalent. 
\begin{corollary}\label{cor_cc_eq}
Let $\e$ be a super-operator on $\h$ with the orthogonal decomposition presented in Eq.(\ref{eq_mini_dec}). 
Then the relation 
$$\{(p, q) : 1\leq p, q\leq m,  \l(\h_p,\h_q)  \mbox{ is continuous}\}$$
 is an equivalence relation. That is,
  for any $p$, $q$, and $r$, \begin{itemize}
    \item [(1)] (reflexivity) $\l(\h_p,\h_p)$ is continuous;
    \item [(2)] (symmetry) if $\l(\h_p,\h_q)$ is continuous, then so is $\l(\h_q,\h_p)$;
    \item [(3)] (transitivity) if $\l(\h_p,\h_q)$ and $\l(\h_q,\h_r)$ are both continuous, then so is $\l(\h_p,\h_r)$.
      \end{itemize}
\end{corollary}


With Corollary~\ref{cor_cc_eq}, we can group together minimal subspaces by continuous coherences and obtain a structure of $fix(\e)$, in a similar way to Theorem~\ref{Theo_NS_Dec} for stationary coherences. 

\begin{theorem}\label{Theo_CC_dec}
  Let $\e$ be a super-operator on $\h$. There is a unique  orthogonal decomposition of $\h$
  \begin{eqnarray}\label{Eq_unique}
    \h=\bigoplus_{l=1}^{L}\x_l\bigoplus \k.
  \end{eqnarray}
  where \begin{itemize}
    \item[(1)] $\k$ is the largest transient subspace;
    \item[(2)] each $\x_l$ is either a minimal subspace or can be further decomposed into mutually orthogonal minimal subspaces with continuous coherences between any two of them:
    \begin{eqnarray}\label{Eq_dec_CC}
      \x_l=\bigoplus_{p=1}^{m_l}\b_{l,p}=\mathbb{C}^{m_l}\bigotimes \b_l,  \ \ \b_l\simeq \b_{l,p} \ \ \forall p
    \end{eqnarray}  such that  the Kraus operators $\{E_k\}$ of $\e$ have a corresponding block form:
      \begin{eqnarray}\label{Eq_CC_dec_Kraus}
    \renewcommand{\arraystretch}{1.2}
E_k=\left[\begin{array}{c|c}
  \begin{array}{ccc}
  U_{1}\otimes E_{k,1} &  &\\
   & \ddots &\\
   & & U_{L}\otimes E_{k,L}  
  \end{array} & T_k\\
  \hline
0&K_k
\end{array}\right]
  \end{eqnarray}
for some operators $E_{k,l}\in \l(\b_l)$, $K_k\in \l(\k)$, $T_k\in\l(\k,\k^\perp)$, and    unitary matrix $U_l=diag(e^{i\theta_{l,1}},\cdots,e^{i\theta_{l,m_l}})$ for some real numbers $\{\theta_{l,p}\}_{p=1}^{m_l}$ on $\mathbb{C}^{m_l}$. Moreover, $\b_l$ is irreducible under $\e_{l}=\sum_{k}E_{k,l}\cdot E_{k,l}^\dagger$. 
 Furthermore, 
$$fix(\e)=\bigoplus_l[fix(\u_{l})\otimes \rho_l]\bigoplus 0_\k$$ 
where $\u_l(\cdot)=U_l\cdot U_l^\dagger$. 

    \item[(3)] there is no continuous coherence between any minimal subspaces $\b_{l, p}$ and $\b_{l', p'}$ whenever $l\neq l'.$ 
      \end{itemize}
\end{theorem}
{\it Proof.} By Theorem~\ref{Theo_NS_Dec}, there is a unique  orthogonal decomposition of $\h$ as
  $$\h=\bigoplus_{l=1}^{L'}\x'_l\bigoplus \k$$
such that for any orthogonal minimal subspaces $\h_1$ and $\h_2$, they have stationary coherences if and only if $\h_{1}\bigoplus \h_2\in \x'_l$ for some $l$.
      Then we divide $\{\x'_l\}$ into finite disjoint subsets by continuous coherences; that  is for any $l_1\not =l_2$, if there is a continuous coherence between any minimal subspaces in $\x'_{l_1}$ and $\x'_{l_2}$, then they are in the same subset. This can be done as the existence of continuous coherences is an equivalence relation by Corollary~\ref{cor_cc_eq}. Then we define $\{\x_l\}_{l=1}^{L}$ be the set of the direct sum  of all elements in each subset.  Therefore, $\h$ can be uniquely decomposed as $\h=\bigoplus_l\x_l\bigoplus \k.$ Obviously, for any two orthogonal  minimal subspaces $\b_{l_1}\in\x_{l_1}$ and $\b_{l_2}\in\x_{l_2}$, $\l(\b_{l_1},\b_{l_2})$ is continuous if and only if $l_1=l_2$.

Furthermore, for each $l$, $\x_l$ can be further decomposed to mutually orthogonal minimal subspaces:
$$\x_l=\bigoplus_{p=1}^{m_l}\b_{l,p}$$
By Lemma~\ref{lem_block_eq}, %we have that $\{\e_{l,p}\}_p$ is a set of quantum operation which are unitarily equivalent to each other. So 
in  an appropriate decomposition  of $\x_l=\bigoplus_{p=1}^{m_l}\b_{l,p}=\mathbb{C}^{m_l}\bigotimes \b_l$ and $\b_l\simeq \b_{l,p}$ for all $p$:
\begin{eqnarray}
    \renewcommand{\arraystretch}{1.2}
E_k=\left[\begin{array}{c|c}
  \begin{array}{ccc}
  U_{1}\otimes E_{k,1} &  &\\
   & \ddots &\\
   & & U_{L}\otimes E_{k,L}  
  \end{array} & T_k\\
  \hline
0&K_k
\end{array}\right]
  \end{eqnarray}
and $\b_l$ is irreducible under $\e_{l}:=\sum_{k}E_{k,l}\cdot E_{k,l}^\dagger$  for all $l$, 
where $U_l=diag(e^{i\theta_{l,1}},\cdots,e^{i\theta_{l,m_l}})$ for a set of real numbers $\{\theta_{l,p}\}_{p=1}^{m_l}.$  From Theorem~\ref{Theo_NS_Dec} and noting that stationary coherences is continuous, we have that 
$$fix(\e)=\bigoplus_l[fix(\u_{l})\otimes \rho_l]\bigoplus 0_\k.$$ 
where $\rho_l$ is the unique stationary state of $\e_l.$ \hfill $\Box$
\begin{corollary}\label{Cor_mini}
  Let $\e$ be a super-operator  on $\h$. In a decomposition of $\h$ in Eq.(\ref{Eq_dec_CC}) :
$$\h=\bigoplus_l(\mathbb{C}^{m_l}\bigotimes \b_l)\bigoplus \k,$$ for any minimal subspace $\h'$, there is a pure state $|\psi\rangle\in \mathbb{C}^{m_l}$ for some $l$ such that $\h'=\textrm{supp}(|\psi\rangle\langle \psi|)\otimes \b_l$. 
\end{corollary}

%Let $A\in\b(\h)$ and $\e(A)=e^{i\theta}A$ for some $\theta$. Then $A\in B(\k^\perp)$ by \cite[Lemma 2]{guan2017super}. From the definition of continuous coherences, we have that $A=\bigoplus_l \bar{A}_l$ with $\in B(\mathbb{C}^m_l\bigotimes \b_l)$,  and $\sum_{k}U_l\otimes E_{k,l} \bar{A}_l U_l^\dagger\otimes E_{k,l}^\dagger=e^{i\theta}\bar{A}_l$. As $U_l$ is diagonal, $\bar{A}_l=$ 


 The above theorem shows  minimal subspaces with continuous coherences can construct  decoherence-free subsystem $\mathbb{C}^{m_l}$. Fortunately, we can further show that others are subspaces of the decoherence-free subsystems constructed in Eq.$(\ref{Eq_dec_CC})$.
 \begin{theorem}\label{Theo_find_DFS}
   Let $\e$ be a super-operator  on $\h$ with the unique decomposition:
$$\h=\bigoplus_l(\mathbb{C}^{m_l}\bigotimes \b_l)\bigoplus \k,$$
presented in Theorem~\ref{Theo_CC_dec}. Then 
    subsystem $\h_A$ is decoherence-free if and only if \begin{itemize}
      \item [(1)] $\h_A\subseteq\mathbb{C}^{m_l}$ for some $l$;
      \item [(2)] $\h_A$ is a support of some stationary state of $\u_l(\cdot)=U_{l}\cdot U_l^\dagger$,
    \end{itemize} where $U_l$ is the corresponding unitary matrix on $\mathbb{C}^{m_l}$ in the decomposition Eq.(\ref{Eq_CC_dec_Kraus}).  
 
 \end{theorem}
 {\it Proof.} Assume that $\h_A$ is decoherence-free. By Theorem~\ref{Theo_CC_dec} and Corollary~\ref{Cor_mini}, $\h_A\subseteq \mathbb{C}^{m_l}$ for some $l.$ From the definition of decoherence-free subsystems and the restriction of $\e$ onto $\mathbb{C}^{m_l}$ being  $\u_l$, $\h_A$ is a decoherence-free subspace under $\u_l$ and $\u(P_A)=P_A$, where $P_A$ be the projector onto $\h_A$. 

To prove the other direction,  we observe that if  $\h_A$ is a support of some stationary state of $\u_l$, then $P_AU_{l}P_A=P_AU_l=U_lP_A$. Thus $\h_A$ is a decoherence-free subspace under $\u_l$. The rest of the proof is direct from Theorem~\ref{Theo_CC_dec}. $\hfill$ $\Box$

This  theorem confirms that the set of decoherence-free subsystems $\{\mathbb{C}^{m_l}\}_l$ identified in Theorem~\ref{Theo_CC_dec} is  optimal; that is any other decoherence-free subsystem is a subspace of one of them. So we only need to implement the decompositions in Theorem~\ref{Theo_CC_dec} and all decoherence-free subsystems can  be easily found by Theorem~\ref{Theo_find_DFS}.

One easy way of achieving this is to first transform all continuous coherences to stationary ones, and then use NSDecompose($\h,\e$) proposed already in the literature. 

For any two operators $E_{k,p}$ and $E_{k,q}$ in Eq.(\ref{Eq_CC_dec_Kraus}) of Theorem~\ref{Theo_CC_dec}, if they are unitarily equivalent with a phase $\theta$, i.e. $E_{k,p}\simeq e^{i\theta}E_{k,q}$, then 
 let $E'_{k,q}=e^{i\theta}E_{k,q}$ and $E'_{k,p}=E_{k,p}$. Then $E'_{k,p}$ is unitarily equivalent to $E'_{k,q}$; a continuous coherence is transformed to be stationary. Using this method, we develop Algorithm 1 to implement decompositions in Theorem~\ref{Theo_CC_dec} and the time complexity is $O(n^8)$, where dim$(\h)=n.$ 
 \begin{algorithm}
\caption{Decompose($\h,\e$)}
\label{Irreducibility}
\begin{algorithmic}
\REQUIRE A Hilbert space $\h$ and a super-operator $\e$ with Kraus operators $\{E_{k}\}_{k=1}^{d}$ on it \\
\ENSURE The two-level decomposition of $\h$ in the form of Eqs.(\ref{Eq_unique}) and (\ref{Eq_dec_CC}). \\

\STATE 
$\{\mathbb{C}^{m_l}\}_{l=1}^L, \{\b_l\}_{l=1}^{L}, \k \leftarrow\textrm{NSDecompose}(\h,\e)$\\
\STATE $\left\{\left[\begin{matrix}
  \oplus_{l=1}^{L} I_{l}\otimes E_{k,l} & T_k\\
0&K_k
\end{matrix}\right]\right\}_{k=1}^{d}\leftarrow\textrm{NSDecompose}(\h,\e)$ \\

\STATE $\l\leftarrow \{1,2,\cdots,L\}$\\
\FOR{each $p\leftarrow 1 \textrm{ to }L$ }
\IF{$p\in \l$}
\FOR{each $q\leftarrow p+1 \textrm{ to }L$ with $q\in \l$}
\STATE$M\leftarrow\sum_{k}E_{k,p}\otimes E_{k,q}^{*}$
\IF{$\lambda(M)$ has one element with magnitude one}
\STATE $\theta\leftarrow$ $\textrm{tr}(E_{k,p})/\textrm{tr}(E_{k,q})$  
\STATE $E_{k,q}\leftarrow e^{i\theta}E_{k,q}$
\STATE $\l\leftarrow \l\setminus\{q\}$
\ENDIF
\ENDFOR
\ENDIF
\ENDFOR
\FOR{ each $ k\leftarrow$ $1$ to $d$ }
\STATE $E_k\leftarrow\left[\begin{matrix}
  \oplus_{l=1}^{L} I_{l}\otimes E_{k,l} & T_k\\
0&K_k
\end{matrix}\right]$ 
\ENDFOR
\RETURN NSDecompose($\h,\{E_{k}\}_{k=1}^d$)
\end{algorithmic}
\end{algorithm}

Now we back to see Example 1. By Algorithm 1, we can confirm that the first subsystem $\h_A\simeq\mathbb{C}^4$ is decoherence-free and further compute that the evolution on it is $|0\rangle\langle0|+|1\rangle\langle1|-|2\rangle\langle2|-|3\rangle\langle3|$. Thus we can store 4-qubit information in this subsystem, which doubles the capacity of  noiseless subsystems.  
\section{Imperfect Initialization}
In the last section, we have developed techniques to find all decoherence-free subsystems under perfect initialization. In this section, we plan to find counterparts under imperfect initialization. These subsystems are more useful in experiments. 

In perfect initialization, the co-subsystem of a decoherence-free subsystem is not important, but it plays an essential role when we allow imperfect initialization. Choosing an appropriate co-subsystem for a candidate of decoherence-free subsystems is an natural problem.   


\begin{theorem}\label{Theo_check_FT}
Assume imperfect initialization.   Let $\e$ be a super-operator  on $\h$ with unique decomposition  
$$\h=\bigoplus_l(\mathbb{C}^{m_l}\bigotimes \b_l)\bigoplus \k,$$ 
presented in Eq.(\ref{Eq_dec_CC}). For any $l$ and $\h_A\subseteq \mathbb{C}^{m_l}$, 
if $\h_A$ with co-subsystem $\b_l$ is not decoherence-free, then there are no non-trial co-subsystems for $\h_A$ such that $\h_A$ is decoherence-free, where $\h_B$ a trial co-subsystem if $\h=\h_A\bigotimes\h_B.$ 
\end{theorem}

{\it Proof.}  It is easy to observe that any co-subsystem $\h_B$ (can not to be irreducible) of $\h_A$ must satisfy $\b_{l}\subseteq \h_{B}$ from Corollary~\ref{Cor_mini}.

Obviously, if $\h_A$ is decoherence-free in  $\h=\h_A\bigotimes \h_B\bigoplus\y$ with $\y=(\h_A\bigotimes \h_B)^{\perp}\not= \emptyset$,  then $\h_A$ is also decoherence-free in $\h=\h_A\bigotimes \b_l\bigoplus\k',$ as $\h_A\bigotimes \b_l\subseteq \h_A\bigotimes \h_B$. $\hfill$ $\Box$

The above theorem tells us a fact that the co-subsystems identified in Eq.(\ref{Eq_dec_CC}) are sufficient to determine whether a subsystem is   decoherence-free  or not under imperfect initialization. For example, in Example 1, we can get a decoherence-free subsystem $\h_A$ and the co-subsystem $\h'$ under perfect initialization from Algorithm 1. Then through Theorem~\ref{Theo_check_FT}, 
we can use the co-subsystem $\h'$ to verify that  $\h_A$ is decoherence-free with trial co-subsystem $\h_B$ under imperfect initialization.
 
\section{An application to Basis of Matrix Product States}
Describing quantum many body systems is  not scalable due to the exponential growth of the Hilbert space dimension with the number of subsystems.  Matrix Product States (MPS), a special case of tensor networks (a theoretical and numerical tool describing quantum many-body systems), have proven to be a useful family of quantum states for the description of  ground states of  one-dimensional quantum many-body systems \cite{cirac2017matrix}.  

Given a  tensor $\a=\{A_{k}\in \mathcal{M}_D\}_{k=1}^{d}$ with a Hilbert space $\h_d=\textrm{lin.span}\{|k\rangle\}_{k=1}^{d}$, where $\mathcal{M}_{D}$ denotes $D\times D$ complex matrices, it generates a family of translationally invariant MPS, namely
$$V(\a)=\{|V_{n}(\a)\rangle\}_{n\in \mathbb{N^+}},$$
where $$|V_{n}(\a)\rangle=\sum_{k_1,\cdots,k_n=1}^{d}\textrm{tr}(A_{k_1}\cdots A_{k_n})|k_1\cdots k_{n}\rangle\in\h_d^{\otimes n}$$
Here, each $|V_{n}(\a)\rangle$ corresponds to a state of $n$ spins of physical dimension $d.$
Then we can define an associated completely positive map $\e_\a(\cdot)=\sum_{k=1}^{d}A_{k}\cdot A_{k}^\dagger$.


By \cite{cuevas2017irreducible}, we can always find a set of irreducible tensors $\{\a_{j}\}_{j=1}^{m}$ with the same Hilbert space  $\h_d$, and a set of complex number $\{\mu_{j}\}_{j=1}^{m}$ such that for any $n\in\mathbb{N^{+}}$
\begin{eqnarray}\label{Eq_MPS}
  |V_n(\a)\rangle=\sum_{j=1}^{m}\mu_j^{n}|V_n(\a_{j})\rangle
\end{eqnarray}
where a tenor is called irreducible if the associated map is CPTP and irreducible. That is for any tensor $\a$,  the generated MPS can be linearly represented by MPS of a set of irreducible tensors. Therefore, studying irreducible tensors is an interesting problem, especially the conditions allowing two tensors describing the same family of
MPS.
First, we can group irreducible tensors that are essentially the same in the following sense.
\begin{definition}[\cite{cuevas2017irreducible}]
  We say that two irreducible tensors with the same Hilbert space  $\h_d$, say $\a=\{A_{k}\}_{k=1}^{d}$ and $\b=\{B_{k}\}_{k=1}^{d}$, are repeated if there exist a phase $\theta$ and a unitary matrix $U$ so that 
  $$A_{k}=e^{i\theta}UB_kU^\dagger \ \forall k$$
\end{definition}

By the definition, $\a$ and $\b$ are repeated, then $|\a\rangle_{n}=e^{in\theta}|\b\rangle_{n}$ for all $n\in \mathbb{N^+}$. Therefore, for any tensor $\a$, we can reduce the set of irreducible tensors  in Eq.(\ref{Eq_MPS}) to be non-repeated. Such a simplified set is called a basis of $\a$. 

Many results obtained for MPS rely on the basis.  The fundamental problem of MPS is to  relate different tensors rising the same MPS: for any two different tensors $\a$ and $\b$, they can generate the same MPS, i.e. $V(\a)=V(\b)$, which introduces an ambiguity for analyzing many-body states by MPS generated by tensors. This problem can be answered by the basis. That is, if $\a$ and $\b$ have the same MPS, then their basis must be related by a unitary transform \cite{cuevas2017irreducible}. Therefore, determining whether two irreducible tensors are  repeated or not  is a key problem. Even though such repeatability relation can be verified by Jordan decomposition of matrices by the definition, Jordan decomposition is sensitive to errors and should be avoided in numerical analysis. Here we propose a feasible method to achieve this by the results of continuous coherences in previous sections. 
\begin{theorem}
  Given two irreducible tensors with the same Hilbert space $\h_d$, $\a=\{A_{k}\}_{k=1}^{d}$ and $\b=\{B_{k}\}_{k=1}^{d}$, they are repeated if and only if $\lambda(\e_{\a,\b})$ has an element with magnitude one, where $\e_{\a,\b}=\sum_{k=1}^d A_{k}\cdot B_{k}^\dagger.$
\end{theorem}
{\it Proof.} Let $\h_\a$ and $\h_\b$ be the corresponding Hilbert spaces of tensors $\a$ and $\b$, respectively; that is $A_k\in \l(\h_\a)$ and $B_k\in \l(\h_\b)$ for all $k$.  Then  the  Hilbert space of $\e$ is $\h_\a\bigoplus \h_\b$, where $\e$ is a super-operator with Kraus operators $\{diag(A_k,B_k)\}_{k=1}^d$. Obviously, $\h_\a$ and $\h_\b$ are both minimal subspaces under $\e$. Then the result is direct from Lemma~\ref{lem_block_eq}. 

\hfill $\Box$

By the above theorem, repeatability  can be easily checked by computing eigenvalues of a linear map, which is a linear algebra exercise.  
\section{Conclusion}
In this paper, we established a structure theory for decoherence-free subsystems. Consequently, an algorithm for finding an optimal set of decoherence-free subsystems under perfect initialization has been developed. Then we gave a simple way to pick up the faulty-tolerant decoherence-free subsystems. After that, these results helped us find a basis for any tensor by computing the eigenvalues of some constructed linear maps. 

For future studies, an immediate topic is to generalize our results to continuous-time quantum systems. In \cite{ticozzi2008quantum}, authors  studied this in quantum control setting and expected to obtain a linear-algebraic approach for finding all decoherence-free subsystems  for a given generator, describing the system.
\section*{Acknowledgment}


The authors would like to thank...


% Can use something like this to put references on a page
% by themselves when using endfloat and the captionsoff option.
\bibliographystyle{IEEEtran}
% argument is your BibTeX string definitions and bibliography database(s)
\bibliography{bib}




% trigger a \newpage just before the given reference
% number - used to balance the columns on the last page
% adjust value as needed - may need to be readjusted if
% the document is modified later
%\IEEEtriggeratref{8}
% The "triggered" command can be changed if desired:
%\IEEEtriggercmd{\enlargethispage{-5in}}

% references section

% can use a bibliography generated by BibTeX as a .bbl file
% BibTeX documentation can be easily obtained at:
% http://mirror.ctan.org/biblio/bibtex/contrib/doc/
% The IEEEtran BibTeX style support page is at:
% http://www.michaelshell.org/tex/ieeetran/bibtex/
%\bibliographystyle{IEEEtran}
% argument is your BibTeX string definitions and bibliography database(s)
%\bibliography{IEEEabrv,../bib/paper}
%
% <OR> manually copy in the resultant .bbl file
% set second argument of \begin to the number of references
% (used to reserve space for the reference number labels box)


% biography section
% 
% If you have an EPS/PDF photo (graphicx package needed) extra braces are
% needed around the contents of the optional argument to biography to prevent
% the LaTeX parser from getting confused when it sees the complicated
% \includegraphics command within an optional argument. (You could create
% your own custom macro containing the \includegraphics command to make things
% simpler here.)
%\begin{IEEEbiography}[{\includegraphics[width=1in,height=1.25in,clip,keepaspectratio]{mshell}}]{Michael Shell}
% or if you just want to reserve a space for a photo:


\end{document}


